\section{Sportpsychologie}


\begin{question}{3}
Definieren Sie den Begriff der Sportpsychologie und nennen Sie die Ziele der Sportpsychologie im Schneesport.
\end{question}
\begin{solution}
Die Sportpsychologie befasst sich mit dem Verhalten und Erleben im Rahmen sportlicher Aktivit"at. Sie ist darauf gerichtet, dieses Verhalten und Erleben zu beschreiben, zu erklären, zu beeinflussen und das gewonnene Wissen praktisch anzuwenden.\\
\emph{Leistungssport:} Optimierung des Ablaufs sportlicher (H"ochst-)Leistungen im Wettkampf\\
\emph{Breitensport:} Unterstützung sportbezogener Lernprozesse und F"orderung von Spa"s am Schneesport\\\\
\citetitle{theorie} Seite 248
\end{solution}

\begin{question}{4}
Beschreiben Sie Faktoren, die das Lernklima im Unterricht beeinflussen und kennzeichnen Sie die Faktoren f"ur ein positives Lernklima.
\end{question}
\begin{solution}
\emph{Faktoren}
\begin{itemize}
\item Ad"aquate Aufgabenstellung zur Vermeidung von "Uber- oder Unterforderung
\item Angst als Gegner des Lernen (Aufbau von Selbstvertrauen und Umgang mit Angst)
\item Gruppenklima und Gruppenzusammenhalt (gute Stimmung und gegenseitige Unter-
st"utzung)
\item F"uhrungsstil des Skilehrers
\end{itemize}
\emph{Positives Lernklima:} Aufgabenstellung sehr wichtig. Eher aufgabenorientiertes (individuelle Bem"uhen der Sch"uler im Vordergrund), als wettbewerbsorientiertes (Leistungsvergleich im Vordergrund) Klima.\\\\
\citetitle{theorie} Seite 249-250
\end{solution}

\begin{question}{4}
Skizzieren Sie den Prozess der Kommunikation und nennen Sie vier Einflussfaktoren auf diesen.
\end{question}
\begin{solution}
\begin{figure}[H]
  \centering
  \includegraphics[width=12cm]{pic/kommunikation.jpg}
  \label{fig:kommunikation}
\end{figure}
\emph{Einflussfaktoren:}
\begin{itemize}
\item Aktuelle Stimmung und Emotionen
\item Erwartungen von Sender und Empfänger
\item Das gemeinsame Wissen (Verstehen Sender und Empfänger die gleichen Dinge unter den gleichen Begriffen?)
\item Aufmerksamkeit und Interesse
\item Zuf"allige Umst"ande
\item St"orungen wie z.B. Wind oder Betrieb auf der Piste
\end{itemize}
\citetitle{theorie} Seite 251-252
\end{solution}

\begin{question}{6}
Beschreiben Sie das Vier-Ohren-Modell der Kommunikation anhand eines Beispiels aus dem Schneesport.
\end{question}
\begin{solution}
\begin{figure}[H]
  \centering
  \includegraphics[width=12cm]{pic/vierohren.jpg}
  \label{fig:vierohren}
\end{figure}
\begin{figure}[H]
  \centering
  \includegraphics[width=12cm]{pic/vierohrenbsp.jpg}
  \label{fig:viewrohrenbsp}
\end{figure}
\citetitle{theorie} Seite 252-253
\end{solution}

\begin{question}{6}
Als Schneesportlehrer müssen Sie f"ur eine erfolgreiche Kommunikation aktiv Zuh"oren. Wie setzen Sie dies konkret im Unterricht um?
\end{question}
\begin{solution}
\begin{figure}[H]
  \centering
  \includegraphics[width=12cm]{pic/zuhoeren.jpg}
  \label{fig:zuhoeren}
\end{figure}
\citetitle{theorie} Seite 255
\end{solution}

\begin{question}{6}
Wie geben Sie Feedback an Ihre Sch"uler? Beschreiben Sie in diesem Zusammenhang f"unf Feedbackregeln.
\end{question}
\begin{solution}
\emph{Feedbackregeln}
\begin{itemize}
\item Als pers"onliche Wahrnehmung in Ich-Form: Ich habe gesehen, dass du rhythmisch und fließend gefahren bist., statt: Du fährst rhythmisch und fließend
\item Situationsbezogen: Bei dieser Fahrt hast du den Außenski wenig belastet., statt: Deine Außenskibelastung ist immer zu schwach.
\item Beschreibend ohne Wertung: Gerade bist du in R"uckenlage gefahren, statt: Du h"angst hinten drin, das ist schlecht.
\item Bezogen auf ver"anderbares Verhalten. Kritisiert wird nur das Verhalten, nicht die Person: Du sprichst zu laut., statt Du bist zu laut.
\item An den Bed"urfnissen des Sch"ulers ausgerichtet: f"ur Einsteiger h"aufiger, f"ur Fortgeschrittene seltener
\item Immer mit einem positiven Aspekt beginnen.
\end{itemize}
\citetitle{theorie} Seite 257
\end{solution}

\begin{question}{5}
Warum k"onnen zwischen Lehrer und Sch"uler Probleme bei der Kommunikation auftreten? Welche Ursachen k"onnen hierf"ur grunds"atzlich vorhanden sein?
\end{question}
\begin{solution}
\emph{Zwischen Kan"alen:} Wenn Lehrer und Sch"uler auf unterschiedlichen Kan"alen senden und Empfangen.\\
\emph{Innerhalb eines Kanals:} Beeintr"achtigungen auf Beziehungsebene (Streit)\\
St"orungen von Außen, Aufmerksamkeit, Interesse, Unterschiedliche Ausdrucksweise oder Sprache\\\\
\citetitle{theorie} Seite 252-257
\end{solution}

\begin{question}{4}
Definieren Sie den Begriff der Selbstwirksamkeit. Wie k"onnen Sie das Selbstvertrauen ihrer Sch"uler st"arken? Geben Sie dazu drei Beispiele / praktische Tipps.
\end{question}
\begin{solution}
\emph{Selbstwirksamkeit:} Selbstwirksamkeit ist der Glaube an die eigenen F"ahigkeiten, eine bestimmte Aufgabe meistern zu k"onnen. Sie ist somit Aufgaben- und Situationsspezifisch und kann trainiert werden.\\
\emph{Stärkung der Selbstwirksamkeit:} 
\begin{itemize}
\item Pers"onliche Erfahrungen: Wiederholte Erfolgserlebnisse f"ordern das Selbstwirksamkeitserleben im Sport am besten.
\item Stellvertretende Erfahrungen: Das Betrachten eines anderen w"ahrend der Bewegungsausf"uhrung steigert die Selbstwirksamkeitserwartung, und zwar umso mehr, je "ahnlicher sich beide Personen in aufgabenrelevanten Eigenschaften sind. Auch die Imagination der eigenen Bewegung kann eine Quelle von mehr Selbstwirksamkeitserleben sein.
\item Verbale "Uberzeugung: Positive, auffordernde S"atze des Trainers, die dem Sportler Mut machen. Der Sportler kann auch lernen, sich selbst gut zuzureden, also sich selbst positive Instruktionen zu geben.
\item Emotionales Arousal: In welcher Gef"uhlslage wir uns befinden und v.a., wie wir dieses Gef"uhl (z.B. Angst) und die damit verbundenen k"orperlichen Zust"ande (z.B. Herzklopfen, Zittern) bewerten, kann sich stark auf unser Selbstvertrauen auswirken.
\end{itemize}
\citetitle{theorie} Seite 264-266
\end{solution}

\begin{question}{6}
Beschreiben Sie den Teufelskreis der Angst.
\end{question}
\begin{solution}
\begin{figure}[H]
  \centering
  \includegraphics[width=12cm]{pic/angst.jpg}
  \label{fig:angst}
\end{figure}
Was passiert, wenn jemand keine Selbstwirksamkeitserwartung aufbaut und sich z. B. steile Abfahrten, Tiefschneefahrten oder Wettk"ampfe nicht zutraut? Angst macht sich breit und das bereits Gelernte ist viel schlechter abrufbar. Der Sch"uler wird die Aufgabe nicht so gut meistern, wie er es aufgrund seiner technischen Fertigkeiten k"onnte. Am Teufelskreis der Angst von Margraf und Schneider (1990) wird deutlich, wie sehr Gef"uhle. Gedanken und Verhalten zusammenh"angen und sich aufschaukeln k"onnen und wo man eingreifen kann, um der lch-kann-nicht-Falle zu entkommen.\\\\
\citetitle{theorie} Seite 267
\end{solution}

\begin{question}{2}
Nennen Sie vier Entspannungsverfahren, die in der Sportpsychologie angewendet werden.
\end{question}
\begin{solution}
\begin{itemize}
\item Autogenes Training
\item Progressive Muskelentspannung
\item Atemtechniken
\item Meditation
\end{itemize}
Quelle nicht gefunden.
\end{solution}