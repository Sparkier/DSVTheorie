\section{Sportorganisation}

\begin{question}{1}
Mit welcher Qualifikation ist man berechtigt, eine DSV-Skischule zu f"uhren? Nennen sie alle Qualifikationsm"oglichkeiten.
\end{question}
\begin{solution}
\begin{itemize}
\item DSV-Skilehrer
\item Instructor mit Skischulleiter Ausbildung
\end{itemize}
Quelle nicht gefunden.
\end{solution}

\begin{question}{2}
Welche Voraussetzungen m"ussen erf"ullt sein, um als Skischule den Status DSV-Skischule zu erhalten?
\end{question}
\begin{solution}
\begin{itemize}
\item Erf"ullung des Kriterienkatalogs des DSV und der Landesskiverb"ande
\item Regelm"a"sige Fortbildung der Skischulleiter alle 2 Jahre
\end{itemize}
\citetitle{theorie} Seite 93
\end{solution}

\begin{question}{5}
Geben Sie einen "Uberblick zur IVSI. Gehen Sie dabei auf die Bedeutung f"ur die Ausbildung im DSV ein.
\end{question}
\begin{solution}
\begin{itemize}
\item IVSI = Internationaler Verband der Schneesport-Instruktoren
\item Zusammenschluss nicht professioneller, aber ausgebildeter und gepr"ufter Lehrkr"afte in den Disziplinen Ski-Alpin, Snowboard, Skitour, Telemark und Nordic.
\item Die Ausbildungsgrundlage stellen die Mindestanforderungen f"ur die Ausbildung in den Mitgliedsverb"anden.
\item Die IVSI-Marke ist eine international einheitliche Marke f"ur den Instruktoren-Ausweis und dient als Nachweis f"ur eine qualifizierte Aus- und Fortbildung. Auch der DSV vergibt die IVSI-Marke.
\end{itemize}
\citetitle{theorie} Seite 53
\end{solution}

\begin{question}{4}
Skizzieren Sie die nationale und internationale Struktur des Schneesports.
\end{question}
\begin{solution}
Siehe Tabelle \ref{strukturschneesport}.\\\\
\begin{table}
\caption{Nationale und internationale Struktur des Schneesports.}
  \label{strukturschneesport}
  \scriptsize
  \begin{center}
    \begin{tabular}{p{2,5cm}|p{5,5cm}|p{5,5cm}}
       & \textbf{Fachlich} & \textbf{"Uberfachlich}\\
    \hline
      International & Interski International (IVSS, IVSI, ISIA), IBU, FIS & IOC\\
      National (Bund) & Interski Deutschland (DVS) & DOSB, DSH
    \end{tabular}
  \end{center}
\end{table}
\citetitle{theorie} Seite 44
\end{solution}

\begin{question}{4}
Welche Ausbildungsm"oglichkeiten bietet der DSV seinen Mitgliedern im Leistungs- und Breitensport?
\end{question}
\begin{solution}
Siehe Tabelle \ref{ausbildungen}.\\\\
\begin{table}
\caption{Nationale und internationale Struktur des Schneesports.}
  \label{ausbildungen}
  \scriptsize
  \begin{center}
    \begin{tabular}{p{6,75cm}|p{6,75cm}}
       \textbf{Leistungssport} & \textbf{Breitensport}\\
    \hline
      Trainer A, B, C, Sportwissenschaftler (in Kooperation mit der Uni Leipzig), Mit Trainer A Diplomtrainer m"oglich, IHK Wirtschafts-/Sportfachwirt & Trainer A, B, C, Skischulleiter, IHK Wirtschafts-/Sportfachwirt
    \end{tabular}
  \end{center}
\end{table}
\citetitle{theorie} Seite 66-67
\end{solution}

\begin{question}{5}
Geben Sie einen "Uberblick zu INTERSKI DEUTSCHLAND (DVS) und nennen Sie f"unf Mitgliedsverb"ande.
\end{question}
\begin{solution}
Dachverband derjenigen  Mitgliedsverb"ande, die sich mit Unterricht und Ausbildung im Schneesport befassen. Dazu z"ahlen: Berufsverb"ande, Amateurverb"ande sowie sonstige Verb"ande, Organisationen und Beh"orden, zu deren Aufgaben die Aus-, Weter- und Fortbildung im Schneesport geh"oren. Mitglieder sind:
\begin{itemize}
\item Deutscher Skiverband (DSV)
\item Deutscher Skilehrerverband (DSLV)
\item Deutscher Alpenverein (DAV)
\item Deutscher Turnerbund (DTB)
\item Naturfreunde Deutschlands
\item Arbeitsgemeinschaft Schneesport an Hochschulen (ASH)
\item Bundesministerium f"ur Verteidigung f"ur den Bereich der Bundeswehr
\item Bayrisches Polizeisportkuratorium
\item Deutscher Behindertensportverbad (DBS)
\end{itemize}
\citetitle{theorie} Seite 68-70
\end{solution}

\begin{question}{3}
Beschreiben Sie die Struktur des Deutschen Skiverbandes e.V. und der DSV Gesellschaften.
\end{question}
\begin{solution}
Der DSV ist die Muttergesellschaft dreier unter seinem Dach gegr"undeter GmbHs. Der DSV e.V. gew"ahrleistet den gemeinn"utzigen Teil der gesamten Vereinsarbeit. Alle wirtschaftlichen und vermarktungsrelevanten Aktivit"aten sind ausgegliedert und den drei Gesellschaften Leistungssport, Marketing und Verwaltung zugeordnet.\\
Die drei GmbHs sind durch Lizenz-, Leistungs- und Mietvertr"age mit dem DSV als Muttergesellschaft eng verbunden. Sie sind untereinander durch Agentur- und Leistungsvertr"age verkn"upft und werden jeweils von Aufsichtsr"aten und Gesch"aftsf"uhrern geleitet und verwaltet. Der DSV-Pr"asident hat jeweils den Vorsitz in den Aufsichtsr"aten der Leistungssport und Marketing GmbH. Der DSV-Schatzmeister ist in allen drei Aufsichtsr"aten t"atig und zugleich Aufsichtsratsvorsitzender der Verwaltungs GmbH.\\\\
\citetitle{theorie} Seite 75
\end{solution}

\begin{question}{5}
Was ist das oberste Verbandsorgan im DSV? Beschreiben Sie dessen Mitglieder und Aufgaben.
\end{question}
\begin{solution}
Oberstes Verbandsorgan: Verbandsversammlung\\
Mitglieder:
\begin{itemize}
\item Ordentliche und au"serordentliche Mitglieder des DSV
\item Pr"asidium
\item Vorsitzende der Ausch"usse und Referate
\item Fachreferenten
\item Gesch"aftsf"uhrer der drei GmbHs
\item DSV-Direktoren
\item Sportliche Leiter
\end{itemize}
Aufgaben:
\begin{itemize}
\item Entgegennahme der Berichte des Pr"asidiums
\item Genehmigung des Stellenplans f"u Ehren- und Hauptamt
\item "Anderung und Beschluss aller Ordnungen und der Satzung
\end{itemize}
\citetitle{theorie} Seite 80-81
\end{solution}

\begin{question}{5}
Was ist das oberste Gremium des Bereichs DSV-Sportentwicklung? Beschreiben Sie dessen Mitglieder und Aufgaben.
\end{question}
\begin{solution}
Oberstes Gremium: Jahreskonferenz Sportentwicklung\\
Mitglieder:
\begin{itemize}
\item Mitglieder der F"uhrung Sportentwicklung
\item Pr"asidenten der Landesskiverb"ande
\end{itemize}
Aufgaben:
\begin{itemize}
\item Beratung und Beschluss von Grundsatzfragen und Schwerpunkten im Bereich Sportentwicklung
\item Beschl"usse sind verbindliche Grundlage f"ur alle untergeordneten Gremien und die DSV-Gesch"aftsstelle
\end{itemize}
\citetitle{theorie} Seite 83
\end{solution}

\begin{question}{3}
Beschreiben Sie die Struktur der Aus- und Fortbildung von Lehrkr"aften innerhalb des DSV und der Landesskiverb"ande.
\end{question}
\begin{solution}
Auf den Stufen Trainer-C und Trainer-B Breitensport bilden die Landeslehrteams der jeweiligen Landesskiverb"ande aus. Die Ausbildung zum Trainer-A Breitensport erfolgt zentral durch die Bundeslehrteams der verschiedenen Disziplinen. Eine Ausnahme bilden die Disziplinen Telemark und Ski-Inline, deren Lehrkr"afte bereits ab der Stufe Trainer-C Breitensport zentral vom DSV ausgebildet werden.\\\\
\citetitle{theorie} Seite 67
\end{solution}

\begin{question}{5}
Was sind die Landessportb"unde? Erl"autern Sie deren Aufgaben.
\end{question}
\begin{solution}
Ein Landessportbund ist die Gemeinschaft der Sportvereine, Fachverb"ande, Kreissportb"unde, Stadtsportb"unde und Sportinstitutionen eines Bundeslandes.\\
Aufgaben sind:
\begin{itemize}
\item Interessenvertretung gegen"uber Land, Kommunen und "Offentlichkeit
\item Sportart"ubergreifende F"orderung der Aus- und Fortbildung von "Ubungsleitern, Vereinsmanagern und Jugendleitern
\item Bezuschussung der ehrenamtlichen "Ubungsleiter und Trainer
\item Unterst"utzung der ehrenamtlichen Vorst"ande in Vereinen und Verb"anden
\item F"orderung des Senioren- und Gesundheitssports
\item Koordination der gemeinsamen Aufgaben in Leistungs- und Breitensport, insbesondere im Kinder- und Jugendsport (Talentf"orderung)
\item Unterst"utzung beim Bau unf Erhalt von Sportst"atten
\item Gew"ahrleistung des Versicherungsschutzes
\item F"orderung und Nutzung der Sportwissenschaft und der sport"arztlichen Betreung
\end{itemize}
\citetitle{theorie} Seite 90
\end{solution}