\section{Sportdidaktik}
\begin{question}{4}
Nennen und beschreiben Sie vier didaktische Prinzipien f"ur die Gestaltung des Schneesportunterrichts.
\end{question}
\begin{solution}
Unterricht sollte:
\begin{itemize}
\item erfahrungsorientiert sein: Wir erkl"aren unseren Sch"ulern Zusammenh"ange nicht nur, sondern bieten ihnen Gelegenheiten, sie auch zu erleben und zu ersp"uren.
\item handlungsorientiert sein: Wir sorgen f"ur Lernsituationen, in denen die Sch"uler selbst t"atig sein k"onnen (z.B. bei der Wahl der Pisten, der Organisation in der Gruppe, dem Umgang mit dem Material usw.)
\item individualisiert sein: Wir gehen auf jeden Sch"uler mit seinen jeweiligen Voraussetzungen und Entwicklungszielen ein.
\item reflexiv sein: Wir sprechen mit den Sch"ulern "uber Erlebtes und regen zum Nachdenken dar"uber an.
\end{itemize}
\citetitle{theorie} Seite 197
\end{solution}

\begin{question}{3}
Welche personalen / materialen Voraussetzungen ber"ucksichtigen Sie im Schneesportunterricht? 
\end{question}
\begin{solution}
\emph{Personale Voraussetzungen:}\\
Individuelle Voraussetzungen der Sch"uler: Alter und Entwicklungsstand; Aktuelles schneesportliches K"onnen und Wissen; Einstellungen und Haltungen in Bezug auf bewegungsorientierte Aktivit"aten; Interessen und W"unsche\\
Voraussetzungen der Gruppe: Wie funktioniert die Gruppe im Zu- sammenspiel/ als Einheit? Beeinflusst Unterrichtsmethoden, Inhalte, Fahrtempo etc.\\
Schneesportlehrer: Eigene Voraussetzungen – Wissen und K"onnen, Einstellungen und Haltungen, Interessen und W"unsche; bestimmen den Unterricht grundlegend mit; Gepr"agt durch Vorerfahrungen, die zu unserem pers"onlichen Verst"andnis vom Schneesport f"uhren und bestimmen, was wir wie an unsere Sch"uler weitergeben wollen.\\
\emph{Materielle Voraussetzungen:}\\
Gel"ande- und Schneeverh"altnisse; Wetterverh"altnisse; Ausr"ustung\\\\
\citetitle{theorie} Seite 199-211
\end{solution}

\begin{question}{8}
Nennen Sie drei Unterrichtsverfahren zur Steuerung im Unterricht und deren Merkmale (mindestens drei). Geben Sie zu einem Verfahren ein praktisches Beispiel aus ihrer Disziplin.
\end{question}
\begin{solution}
\emph{Direkte Steuerung:}\\
Schneesportlehrer als Instrukteur; F"uhrt Regie im Unterricht; W"ahlt Ziele und Inhalte aus; Nimmt Einteilung sinnvoller Lerneinheiten vor; Sorgt f"ur klares, strukturiertes, systematisches Vorgehen; Ber"ucksichtigt individuelle Differenzierung und Selbst"andigkeit der Sch"uler\\
Beispiel: Der Schneesportlehrer stellt mehrere aufeinander aufbauende Aufgaben zum Befahren einer Wellenbahn (Befahren in der Falllinie mit Beugen der Beine auf der Welle, Anfahrt in Schr"agfahrt mit Beugen auf der Welle etc.). Er erl"autert den Sch"ulern Aktions-Funktions- Zusammenh"ange und jeweils denn Sinn der Aufgabe. Er beobachtet die Sch"ulerfahrten im Umlaufbetrieb und gibt jedem Sch"uler nach seiner Fahrt ein Feedback zum Bewegungsablauf.\\
\emph{Kooperative Steuerung:}\\
Ziele und Inhalte werden nicht endg"ultig vom Schneesportlehrer festgelegt; Lehrer bringt Themen und Inhalte ein, die verhandelbar sind; Sch"uler handeln aus Forscherdrang, Neugier, Interesse und Motivation durch selbst initiiertes Probleml"osen; Lehrer sorgt f"ur Rahmen und Gelegenheiten, das Gelernte und Erfahrene selbstst"andig und vielseitig einzusetzen\\
Beispiel: Der Schneesportlehrer gibt das Thema bekannt und l"asst die Sch"uler in einer Fahrt in der Falllinie der Wellenbahn testen. Die Sch"uler bringen Ideen ein, welche Bewegungen zum Befahren sinnvoll sein k"onnten. Die Ideen werden erprobt und die Effekte verschiedener L"osungen kurz besprochen. In weiteren Fahrten werden auf Vorschlag des Lehrers und/oder Sch"ulers weitere Aspekte ver"andert, z.B. der Anfahrtswinkel auf die Welle oder das Fahrtempo. Ein kurzes Abschlussgespr"ach fasst die Bewegungserlebnisse zusammen.\\
\emph{Indirekte Steuerung:}\\
Schneesportlehrer gestaltet Lernumgebungen anhand von Gel"andewahl, Materialien und Aufgaben; Die Sch"uler lernen anhand authentischer Probleme un verschiedenen Kontexten; Die Sch"uler lernen im sozialen Austausch\\
Beispiel: Der Schneesportlehrer l"asst die Sch"uler eine Wellenbahn im Umlaufbetrieb befahren. Er ermuntert zu testen, wie es am besten geht bzw. welche unterschiedlichen Bewegungsm"oglichkeiten es gibt. Nach einigen Fahrten l"asst er die Sch"uler mit einem Partner Trainingsteams bilden. Die Paare sprechen bei jedem Durchgang ab, was sie nun ausprobieren wollen. Nach einigen Fahrten stellt der Lehrer Richtungstore auf die Wellen, die zu umfahren sind. Etwas sp"ater ver"andert er deren Position noch weiter aus der Falllinie heraus. Zum Abschluss werden die Varianten und die resultierenden Bewegungserlebnisse besprochen.\\\\
\citetitle{theorie} Seite 218-219
\end{solution}

\begin{question}{4}
Beschreiben Sie jeweils ein Beispiel f"ur eine M"undliche und Schriftliche Reflexion der Unterrichtsinhalte. 
\end{question}
\begin{solution}
\emph{Mündliche Reflexion:} Drei Smileys in den Schnee legen und dann Fragen zu den verschiedenen Ereignissen des Tages stellen. Die Sch"uler laufen bei jeder Frage zu dem Smiley, der ihrer Antwort entspricht.\\
\emph{Schriftliche Reflexion:} Schüler bekommen zu beginn ein Trainingstagebuch, in das sie verschiedene sachen eintragen können. Bestimmte Inhalte sind vorgegeben (Ziele, Selbsteinschätzzungen).\\\\
\citetitle{theorie} Seite 228-229
\end{solution}

\begin{question}{4}
Welche Aspekte der verbalen und nonverbalen Kommunikation ber"ucksichtigen Sie im Schneesportunterricht in der Altersstufe 4-7 Jahre /7-10 Jahre/ 10-12 Jahre/ 12-15 Jahre/ ab 15 Jahre.
\end{question}
\begin{solution}
\emph{Ab 4:}
\begin{itemize}
\item Kinder reagieren stark auf K"orpersprache. Nonverbales ist oft wichtiger als Worte. Al- so: Viel Wert auf Mimik und Gestik legen!
\item Blickkontakt w"ahrend des Gespr"achs ist in diesem Alter nicht so wichtig wie in ande- ren Altersgruppen. Wir sollten nur die Aufmerksamkeit der Kinder sicherstellen.
\item Die Kinder haben eine sehr kurze Aufmerksamkeitsspanne, deshalb kurze S"atze und einfache W"orter verwenden. Gespr"ache mit einzelnen Kindern sollten nur kurz sein.
\item Bei der Formulierung helfen, wenn den Kindern W"orter fehlen. Aber unbedingt nach- fragen, ob unsere Erg"anzung stimmt!
\item Fragen umformulieren und wiederholen, wenn das Kind eine Frage nicht versteht.
\item Bildhafte Sprache und phantasiereiche (Bewegungs-)Geschichten einsetzen.
\item Auf Augenh"ohe kommunizieren schafft N"ahe.
\end{itemize}
\emph{Ab 7:}
\begin{itemize}
\item Wir verwenden weiterhin bildhafte Sprache und phantasiereiche Geschichten.
\item Durch einen Sprung in der kognitiven Entwicklung (Lesen und Schreiben, abstraktes Denken) wird kindgerechtes Erkl"aren funktionaler Zusammenh"ange m"oglich.
\item Ironische Aussagen werden oft nicht verstanden.
\item Bei der Formulierung helfen, wenn den Kindern W"orter fehlen. Aber unbedingt nach- fragen, ob unsere Erg"anzung stimmt!
\item Materielle Belohnungen (eine besondere Abfahrt aussuchen d"urfen, Gummib"archen
etc.) stehen im Vordergrund, die Bedeutung immaterieller Belohnung (Wertsch"atzung
durch uns und Mitsch"uler) nimmt jedoch zu.
\end{itemize}
\emph{Ab 10:}
\begin{itemize}
\item Konkrete Erkl"arungen sind m"oglich, schwierige W"orter m"ussen wir erl"autern, bildhafte Sprache als Variante bzw. Untermalung einsetzen.
\item Erkl"arungen funktioneller Zusammenh"ange werden verstanden und sind sinnvoll, aber die Freude am Ausprobieren "uberwiegt.
\item Langatmige Erkl"arungen langweilen.
\item Die Meinung der gleichaltrigen ist ein wichtiger Ma"sstab, die soziale Akzeptanz spielt eine gro"se Rolle.
\item Die Bedeutung materieller Belohnungen nimmt ab (wird aber immer noch gesch"atzt), die Bedeutung immaterieller Belohnung (Lob/Anerkennung durch Erwachsene und Gleichaltrige) wird sehr gro"s.
\end{itemize}
\emph{Ab 12:}
\begin{itemize}
\item Die intellektuellen F"ahigkeiten nehmen stark zu, Jugendliche denken "uber vieles nach.
\item Heranf"uhren an die Fachsprache (insbes. im Leistungssport).
\item Fachsprache nutzen, aber mit Alternativbegriffen begleiten.
\item Jugendliche wollen nicht belehrt werden. Fragen stellen und die Gruppe gemeinsam an einer Sache arbeiten lassen, funktioniert besser.
\item Die Betonung auf Neues entdecken und ausprobieren ist hier genauso Gewinn bringend wie bei Kindern, um das Interesse wach zu halten.
\item Jugendliche wollen ernst genommen werden. Wenn wir ihnen eine Frage stellen, soll-
ten wir uns auch f"ur die Antwort interessieren.
\item Anerkennung durch Gleichaltrige steht im Zentrum.
\item Die Kommunikation zwischen Gleichaltrigen ist sehr intensiv, gegen"uber Erwachse-
nen deutlich zur"uckhaltender.
\item K"orpersprache verr"at vieles, was verbal nicht mitgeteilt wird, v.a. Unsicherheiten.
\end{itemize}
\emph{Ab 15:}
\begin{itemize}
\item Wir f"uhren Fachsprache ein und nutzen sie anschlie"send.
\item Unsere Aussagen sollten klar verst"andlich, pr"azise und informativ sein.
\item Aktives Zuh"oren zeigt Verbindlichkeit (non-verbal: z.B. Augenkontakt, Kopfnicken; verbal: z.B. Nachfragen, Kommentieren, Sch"uleraussagen in eigene Ansagen einbeziehen).
\item Wir sind als Experten gefragt und "uberzeugen v.a. durch unsere freundlich vermittelte
Fachkompetenz.
\item Wir betreuen individuell und vermitteln auch das entsprechende Gef"uhl.
\item Nonverbale Botschaften der Sch"uler dienen als wichtige Informationsquelle in Bezug auf Dinge, die im Rahmen der Gruppe aufgrund sozialer Gepflogenheiten evtl. nicht ge"au"sert werden.
\end{itemize}
\citetitle{theorie} Seite 234-236
\end{solution}