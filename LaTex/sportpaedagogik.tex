\section{Sportp"adagogik}
\begin{question}{4}
Diskutieren Sie den Begriff der Handlungsfa"ahigkeit im Schneesport.
\end{question}
\begin{solution}
Handlungsf"ahigkeit im engeren Sinne als praxisbezogener Begriff bezieht sich hier auf die spezifische Handlungsf"ahigkeit im Schneesport. Aufgrund der "au"seren Rahmenbedingungen (Outdoor, Skigebiet als Bewegungsraum u.a.) ben"otigen die Sch"uler ein bestimmtes Ma"s an motorischem K"onnen, konkrete sportartspezifische Kenntnisse (FIS-Regeln, Verhalten in der Gruppe, Orientierung im Gebiet u.a.) sowie die Bereitschaft, das Wissen und K"onnen auch tats"achlich anzuwenden. Der Erwerb von Kompetenzen soll die Sch"uler bef"ahigen, sicher, selbsta"andig und souver"an am Schneesport teilzunehmen.\\
Handlungsf"ahigkeit im weiteren Sinne hei"st nicht nur, die Sportart aus"uben zu k"onnen. Die Sch"uler sollen sich im Schneesport so gut auskennen, dass entscheiden k"onnen, was Schneesport f"ur sie bedeutet. Ziel ist, "uber die konkrete sportliche Aktivit"at hinaus, selbstbestimmt und m"undig zu entscheiden, was f"ur das eigene Leben relevant ist und damit die eigene Identit"at zu st"arken.\\\\
\citetitle{theorie} Seite 144
\end{solution}

\begin{question}{5}
Nennen Sie die grundlegenden Zielsetzungen im Schneesport aus sportp"adagogischer Sicht und erl"autern Sie diese jeweils anhand eines Beispiels. 
\end{question}
\begin{solution}
Ziele im Schneesport: bewegungsbezogenes Lernen und Pers"onlichkeitsentwicklung\\
Bewegungsbezogenes Lernen:
\begin{itemize}
\item Bewegungserfahrungen, die sich einer direkten Belehrung entziehen, in m"oglichst gro"ser Vielfalt zu f"ordern: subjektive Innenansicht der Bewegung
\item Bewegungsk"onnen anhand der sportartspezifischen Strukturen und grundlegenden Technikmerkmale einschlie"slich Kenntnissen "uber Aktions-Funktions-Zusammenh"ange vermitteln: objektive Au"senansicht der Bewegung
\item Bewegungsbezogene Einstellungen und Haltungen f"ordern, die eine intensive und ver- antwortungsvolle Auseinandersetzung mit Gegebenheiten und Herausforderungen im Schneesport unterst"utzen
\end{itemize}
Beispiel: Anhand Bewegungsmerkmale Bewegungsk"onnen verbessern.\\\\
Pers"onlichkeitsentwicklung aus p"adagogischer Sicht:
\begin{itemize}
\item Selbstbestimmung: Die Sch"uler sollen f"ur sich selbst Sinn in bestimmten Aktivit"aten oder situativen Gegebenheiten des Schneesports entdecken und entsprechende Entscheidungen treffen.
\item Mitbestimmung: Die Sch"uler k"onnen ihren eigenen Standpunkt, ihre Interessen und Bed"urfnisse, also ihre eigene Sinnfindung innerhalb der Gruppe kommunizieren und
vertreten.
\item Solidarit"at: Die Sch"uler k"onnen die Sinnfindungen anderer, also deren Standpunkte,
Interessen und Bed"urfnisse, anerkennen. Sie sind also kompromissbereit, zeigen Hilfsbereitschaft und treten aktiv f"ur andere Gruppenmitglieder ein.
\end{itemize}
Beispiel: "au"sern eigener Ideen zu neuen "ubungen (Mitbestimmung).\\\\
\citetitle{theorie} Seite 146-153
\end{solution}

\begin{question}{5}
Erl"autern Sie den Unterschied zwischen Au"sen- und Innensicht von Bewegungen. Diskutieren Sie den Einsatz im Unterricht. 
\end{question}
\begin{solution}
\emph{Objektive Au"senansicht:} Sicht auf die Bewegung anhand ihrer Struktur- und Technikmerkmale. Orientiert sich an den konkreten Merkmalen von Bewegungstechniken. Im Unterricht soll anhand der Bewegungsmerkmale und des Wissens "uber aktionale und funktionale Zusammenh"ange im situativen Bezug das Bewegungsk"onnen des Sch"ulers verbessert werden.\\
\emph{Subjektive Innenansicht:} Sicht des Sch"ulers seines sich Bewegens. Sp"uren von Bewegungserfahrungen (Erfahrungsqualit"aten). Bedingt durch die individuelle Auseinandersetzung mit der Umwelt. Im Unterricht soll das Erfahrungsspektrum erweitert werden.\\
\emph{Fazit:} Man sollte sich nicht zu stark im Unterricht an Technikleitbildern orientieren, jeder Sch"uler soll Technikmerkmale individuell umsetzen und mit seinen eigenen k"orperlichen Voraussetzungen in Einklang bringen. Das Bewegungslernen soll im Wechsel von technikorientiertem Bewegungsk"onnen und Bewegungserleben gestaltet sein.\\\\
\citetitle{theorie} Seite 147-149
\end{solution}

\begin{question}{6}
Nennen Sie die drei F"ahigkeiten zur Pers"onlichkeitsentwicklung und beschreiben Sie diese jeweils anhand eines Beispiels.
\end{question}
\begin{solution}
\emph{Selbstbestimmung:} Die Sch"uler sollen f"ur sich selbst Sinn in bestimmten Aktivit"aten oder situativen Gegebenheiten des Schneesports entdecken und entsprechende Entscheidungen treffen.
\emph{Mitbestimmung:} Die Sch"uler k"onnen ihren eigenen Standpunkt, ihre Interessen und Bed"urfnisse, also ihre eigene Sinnfindung innerhalb der Gruppe kommunizieren und vertreten.
\emph{Solidarit"at:} Die Sch"uler k"onnen die Sinnfindungen anderer, also deren Standpunkte, Interessen und Bed"urfnisse, anerkennen. Sie sind also kompromissbereit, zeigen Hilfsbereitschaft und treten aktiv f"ur andere Gruppenmitglieder ein.\\\\
\citetitle{theorie} Seite 151
\end{solution}

\begin{question}{6}
Was verstehen Sie unter reflexivem Lehren und Lernen?
\end{question}
\begin{solution}
Lernen soll reflexiv begleitet werden, um nachhaltig zu sein. Das Nachdenken "uber das Erlebte und dessen Thematisierung im Unterricht gilt als zentrales Element des Lernprozesses. Die Reflexion kann sich auf motorische Aspekte (Bewegungserlebnisse) und auf das psychische (Emotionen) und soziale Erleben (Erlebnisse im Austausch mit anderen) beziehen.\\
\emph{Schwerpunkte der Reflexion:}
\begin{itemize}
\item Die Selbstbeobachtung und Selbstwahrnehmung in aktuellen Situationen, sodass die Sch"uler f"ur das, was sie erleben, sensibilisiert werden.
\item Das Verst"andnis f"ur Zusammenh"ange und Abl"aufe. Warum passiert das was gerade geschieht?
\item Die Verkn"upfung des Erlebten mit fr"uheren Erfahrungen, um das Ereignis vor dem jeweils eigenen Erfahrungshorizont deuten und einordnen zu k"onnen. Dazu geh"ort auch die Verkn"upfung von neuem Wissen mit bereits bekannten Zusammenh"angen. 
\end{itemize}
\emph{Ziel der Reflexion:} Nachhaltiges Verstehen von Bewegungsph"anomenen ebenso wie von psychischen Aspekten und sozialen Ereignissen: Erproben und Verstehen und Einordnung in den Gesamtzusammenhang.\\\\
\citetitle{theorie} Seite 154
\end{solution}

\begin{question}{6}
Nennen Sie die physischen und psychosozialen Lernvoraussetzungen f"ur Sch"uler im Vorschulalter/ Schulkindalter/ in der Pubert"at/ im Erwachsenenalter.
\end{question}
\begin{solution}
Siehe Tabelle \ref{learning}.\\\\
\citetitle{theorie} Seite 156 - 163
\begin{table}
\caption{Lernvoraussetzungen in verschiedenen Altersstufen.}
  \label{learning}
  \scriptsize
  \begin{center}
    \begin{tabular}{p{2,5cm}|p{5,5cm}|p{5,5cm}}
      \textbf{Lernender} & \textbf{physische Lernvoraussetzungen} & \textbf{psychosoziale Lernvoaussetzungen}\\
    \hline
      Vorschulalter & 3-4 Jahre: Lernen von Bewegungsgrundformen; 4-6 Jahre: Ausdifferenzierung der Bewegungsgrundformen, Kombination von Bewegungen und Verbesserung der koordinativen F"ahigkeiten & Intensives Spiel- und Bewegungsbed"urfnis; Aufmerksamkeitsspanne ausgesprochen kurz; Im sozialen Kontakt noch stark auf sich selbst bezogen und teilweise von Gruppenspielen oder Partneraufgaben "uberfordert\\
      Schulkindalter (7J. bis Pubert"at) & Ver"anderung der K"orperproportionen; Sehr gute Last-Kraft- und Hebelverh"altnisse; Ideale Voraussetzungen f"ur das Bewegungslernen; Rasche Fortschritte in der motorischen Lernf"ahigkeit; goldenes Lernalter & Begeisterung und Entdeckerfreude k"onnen leicht in "ubermut und Unkontrolliertheit umschlagen; Spontanes und intuitives Lernen; Verbesserung der Konzentrationsf"ahigkeit und Reflexionsf"ahigkeit, Steigerung der kognitiven Leistungsf"ahigkeit; Entwicklung im Sozialverhalten: mit
sozialem Kontakt steigt das Empfinden f"ur die eigene Individualit"at\\
Pubert"at (M: 11/12 bis 13/15; J: 12/14 bis 15/17) & Gravierender Wachstumsschub, "anderung der gewohnten Proportionen; St"orung der Bewegungsabl"aufe, somit anpassen und neu lernen; Gewohnte Handlungsabl"aufe funktionieren u.U. nicht mehr, neue sind noch nicht stabilisiert; M"ogl. Stagnation des Bewegungslernens; Gro"se Unterschiede im physischen Entwicklungsstand & Ver"anderung des eigenen K"orpers irritiert; Unklare Rolle im sozialen Kontext: der Kindheit entwachsen, im Erwachsenenalter noch nicht wirklich angekommen, dadurch Verunsicherung, durch die das Erleben der eigenen Kompetenz sowohl in bewegungsbezogener als auch in pers"onlichkeitsbezogener Hinsicht reduziert sein kann; Starke Zunahme der intellektuellen F"ahigkeiten; Werden zunehmend selbst"andig; Teilzeit-Experten\\
Erwachsenenalter (Ab 18/20 Jahren) & Ca. 18-30/35 J.: Harmonisierung der k"orperlichen Proportionen, gleichbleibende K"orperproportionen und relativ konstante motorische Leistungsf"ahigkeit; zweites goldenes Lernalter; hohe physische Leistungsf"ahigkeit; Ab ca. 30/35 J.: Motorische Leistungsminderung & Psychische Ausgeglichenheit; Auseinandersetzung mit abnehmender Leistungsf"ahigkeit
    \end{tabular}
  \end{center}
\end{table}
\end{solution}

\begin{question}{4}
Nennen Sie verschiedene Rollen, die Sie als Schneesportlehrer einnehmen k"onnen. Beschreiben Sie eine davon. 
\end{question}
\begin{solution}
\emph{Rollen des Schneesportlehrers:}
\begin{itemize}
\item Lernbegleiter
\item Vorbild oder Modell
\item Ansprechpartner und Vertrauensperson
\item Organisator
\end{itemize}
\emph{Schneesportlehrer als Vorbild:}
Durch unser Verhalten bestimmen wir in gro"sem Ma"se mit, wie Sch"uler in bestimmten Situationen agieren, nicht zuletzt bedingt durch unseren Expertenstatus.\\
Der Skilehrer pr"asentiert sich permanent mit seinem Fahrk"onnen. Sch"uler lernen auch durch Nachahmung, daher spielt unser Fahrk"onnen f"ur das bewegungsbezogene Lernen eine gro"se Rolle.\\
In Bezug auf ein pers"onlichkeitsbezogenes Lernen orientieren sich unsere Sch"uler h"aufig an unserem Verhalten. Umgang mit Leistungsunterschieden in der Gruppe, Vermittlung von Erfolgserlebnissen, Umgang mit Konflikten.\\
Vorbildfunktion hinsichtlich weiterer handlungsrelevanter Kompetenzen im Schneesport: Verhalten im Skigebiet, Sicherheit etc.\\\\
\citetitle{theorie} Seite 163-172
\end{solution}

\begin{question}{6}
Nennen Sie die drei relevanten Aspekte f"ur das F"uhren von Gruppen und erl"autern Sie diese jeweils anhand eines Beispiels.
\end{question}
\begin{solution}
\emph{Verhalten:}\\
Verbales Verhalten: das gesprochene Wort, die Information, Zahlen, Daten, Fakten\\
Nonverbales Verhalten: Mimik, Gestik, K"orperhaltung, Blickkontakt\\
Paraverbales Verhalten: Lautst"arke, Sprechgeschwindigkeit, Sprachmelodie\\
Nonverbale und paraverbale Signale werden h"oher bewertet als die verbalen Signale.\\
Keine Doppelbotschaften senden, bei denen mehrere Signale im Widerspruch zueinander stehen, sondern klar und in den Signalen eindeutig sein.\\
Das gesprochene Wort muss gelebt werden!\\
\emph{Versuch:}\\
Durch Versuche entwickeln wir ein Gesp"ur f"ur die einzelnen Kursteilnehmer (unterschiedliche Lerntypen, jeweilige Motivation) und k"onnen auch die Ansprache individuell modifizieren.\\
\emph{Erfolg:}\\
Im Schneesport kann Erfolg sehr unterschiedlich aussehen, je nach Motivation und Ziel. Damit ist der Begriff bedeutungsoffen.\\
Wesentliche Aufgabe: Erfolg zusammen mit der Gruppe definieren und damit ein Ziel f"ur alle zu finden.\\\\
\citetitle{theorie} Seite 172-175
\end{solution}

\begin{question}{6}
Wie lautet die Definition von F"uhrung. Geben Sie je zwei Beispiele zur verbalen und nonverbalen F"uhrung von Personen im Unterricht Ihrer Disziplin.
\end{question}
\begin{solution}
F"uhrung ist die Art und Weise, wie ich mich verhalte, wenn ich versuche, andere zum Erfolg zu f"uhren.\\
Verbale F"uhrung: das gesprochene Wort, Zahlen, Daten, Fakten\\
Nonverbale F"uhrung: Mimik, Gestik, K"orperhaltung Blickkontakt\\\\
\citetitle{theorie} Seite 173
\end{solution}

\begin{question}{6}
Welche Phasen der Gruppenentwicklung nach Tuckman gibt es? Beschreiben Sie die Phase des Forming/ Storming/ Norming/ Performing/ Adjourning anhand eines Beispiels und einer Skizze.
\end{question}
\begin{solution}
\emph{Forming (Kennenlernen)}\\
F"ur das Gelingen des Kurses sehr wichtig, denn hier treffen die Teilnehmer mit ihren Werten, Einstellungen und Vorerfahrungen zum ersten Mal aufeinander.\\
Sch"uler sind auf Lehrkraft ausgerichtet und verhalten sich untereinander nett, tasten sich ab und versuchen ihren Platz in der Gruppe zu finden\\
Lehrer:\\
Vorgabe von Richtung und Struktur, Reglementierung der Verhaltensweisen, Regeln und Normen\\
Herstellung eines guten Kontakts zwischen Lehrer und Sch"ulern\\
Provokation und F"orderung der Kommunikation unter Sch"ulern\\ 
\emph{Storming (Machtkampf)}\\
Gruppenmitglieder sind bem"uht, Unterschiede hervorzuheben und sie suchen ihren Platz, ihre Rolle in der Gruppe. Dabei kann es zu Konkurrenz- oder Machtk"ampfen und Konflikten kommen.\\
Lehrer:\\
Klare und konsequente Einforderung der Einhaltung der Regeln\\
Lehrer als Moderator beobachtet und agiert prozessorientiert\\
Unterst"utzt die Sch"uler dabei, ihren Platz und ihre Rolle in der Gruppe zu finden\\
Sieht und wertsch"atzt individuelle Leistungen, somit individuelle Aufmerksamkeit\\
\emph{Norming (Vertrautheit)}\\
Die Beziehung zur Lehrkraft verliert an Bedeutung. Die Gruppe kommt allm"ahlich zum Laufen. Gruppe hat sich auf gemeinsame Regeln und Normen geeinigt. Teilnehmer akzeptieren ihre Rollen und f"ullen sie aus. Das Verhalten des Einzelnen wird damit f"ur die anderen einsch"atzbar und kalkulierbar. Das Vertrauen untereinander w"achst und es besteht die Chance auf ein Wir-Gef"uhl.\\
Lehrer:\\
In den Hintergrund treten und f"ur einen guten Rahmen sorgen\\
Mehr und mehr Verantwortung an die Gruppe abgeben 
\emph{Performing (Differenzierung)}\\
Die Zusammenarbeit der Gruppe verl"auft erfolgreich und ihre Mitglieder unterst"utzen sich gegenseitig.\\
Lehrer:\\
Aufrechterhaltung der vorhandenen positiven Gruppendynamik: Wie ein DJ, der es schafft, die Tanzfl"ache zu f"ullen, braucht der Lehrer nun einen Sp"ursinn daf"ur, was die Gruppe gerade braucht. Wir m"ussen mit verschiedenen "ubungen, Aufgaben etc. das bieten, was die Gruppe gerade will.\\
Moderator und Impulsgeber\\
\emph{Adjourning (Abl"osung)}\\
Teilnehmer orientieren sich neu, weg von der Gruppe\\
Lehrer hat die Aufgabe, den Prozess des Abschieds zu steuern\\\\
\citetitle{theorie} Seite 173 - 186
\end{solution}

\begin{question}{6}
Erl"autern Sie den Unterschied zwischen einer Gruppe und einem Team.
\end{question}
\begin{solution}
Ein Team ist eine Gruppe von Personen mit komplement"aren F"ahigkeiten, die f"ur einen ge- meinsamen Zweck und eine Reihe spezifischer Leistungsziele eingesetzt werden. Mitglieder haben sich verpflichtet, miteinander zu arbeiten, um das gemeinsame Ziel zu erreichen, und tragen Verantwortung f"ur die Ergebnisse des Teams.\\
Als Lehrkr"afte m"ussen wir uns gut "uberlegen, ob wir eine funktionierende Gruppe oder ein Team ben"otigen, um ein bestimmtes Ziel zu erreichen. F"ur einen gew"ohnlichen Skikurs ist es sicherlich schon sehr gut, wenn die Gruppe die Performing-Phase erreicht.\\\\
\citetitle{theorie} Seite 184
\end{solution}

\begin{question}{3}
Nennen Sie 6 Regeln zur F"orderung einer positiven Gruppendynamik.
\end{question}
\begin{solution}
\begin{itemize}
\item Alle Mitglieder haben Anspruch darauf, ernst genommen zu werden.
\item Wir sprechen laut und deutlich, sodass alle uns verstehen k"onnen.
\item Der Ton macht die Musik. Wir pr"ufen, in welcher Stimmung wir vor die Gruppe treten.
\item Wir achten nicht nur auf unsere Sprache, sondern auch auf Mimik und Gestik.
\item Wir signalisieren unseren Gruppenmitgliedern, dass wir zuh"oren und sie verstehen. Dabei verwenden wir von ihnen benutzte Worte, wiederholen ihre Aussagen, erinnern an die morgens beim Start ge"au"serten Erwartungen.
\item Wir helfen den anderen, sagen aber auch, wenn wir selbst Hilfe brauchen.
\item Wir arbeiten mit Abschiedsritualen.
\end{itemize}
\citetitle{theorie} Seite 188
\end{solution}

\begin{question}{4}
Wie zeigen Sie echtes Interesse an den Teilnehmern? Welchen Effekt hat das auf den Teilnehmer?
\end{question}
\begin{solution}
\emph{Interesse Zeigen:}
\begin{itemize}
\item Aufmerksames Zuh"oren
\item Teilnehmer ernst nehmen
\item Keine missverst"andlichen Botschaften senden
\item Neugierig sein und Teilnehmer nach Hobbies, Beruf etc. fragen
\item Richtige Sprach- und Bildebene
\item Lernen, wie die Teilnehmer ticken und welche Ansprache sie brauchen
\item Lernen mit welchen methodischen Grunds"atzen und Verfahren wir die Teilnehmer locken k"onnen
\end{itemize}
\emph{Effekt:}\\
Bessere Gruppendynamik, Bessere Lernatmosph"are, Echtes Interesse vermittelt Vertrautheit - Leichteres Lernen f"ur den Teilnehmer\\\\
\citetitle{theorie} Seite 184
\end{solution}