\section{Bewegungslehre}

\begin{question}{5}
    Nennen Sie die fünf Sinnesorgane der Sensorik und erläutern Sie diese jeweils anhand eines Beispiels.
\end{question}
\begin{solution}
    Frage noch nicht bearbeitet. Bitte melde dich in einer Mail (Mailadresse auf dem Titelblatt) oder nutze git, wenn du eine Lösung für die Frage hast. Alle anderen können dann von einer korrekten Lösung profitieren.
\end{solution}

\begin{question}{6}
    Beschreiben Sie das kybernetische Regelkreismodell. Welche Schlussfolgerungen ziehen Sie daraus für Ihren Unterricht?
\end{question}
\begin{solution}
    Frage noch nicht bearbeitet. Bitte melde dich in einer Mail (Mailadresse auf dem Titelblatt) oder nutze git, wenn du eine Lösung für die Frage hast. Alle anderen können dann von einer korrekten Lösung profitieren.
\end{solution}

\begin{question}{6}
    Beschreiben Sie einen modernen theoretischen Ansatz des Bewegungslernens.
\end{question}
\begin{solution}
    Frage noch nicht bearbeitet. Bitte melde dich in einer Mail (Mailadresse auf dem Titelblatt) oder nutze git, wenn du eine Lösung für die Frage hast. Alle anderen können dann von einer korrekten Lösung profitieren.
\end{solution}

\begin{question}{6}
    Stellen Sie traditionelle und moderne Lehr- und Lernkonzepte gegenüber. Gehen Sie dabei auf Vor- und Nachteile ein.
\end{question}
\begin{solution}
    Frage noch nicht bearbeitet. Bitte melde dich in einer Mail (Mailadresse auf dem Titelblatt) oder nutze git, wenn du eine Lösung für die Frage hast. Alle anderen können dann von einer korrekten Lösung profitieren.
\end{solution}

\begin{question}{4}
    Kategorisieren Sie die Sportarten nach Aufgabentypen. Geben Sie jeweils ein Beispiel.
\end{question}
\begin{solution}
    Frage noch nicht bearbeitet. Bitte melde dich in einer Mail (Mailadresse auf dem Titelblatt) oder nutze git, wenn du eine Lösung für die Frage hast. Alle anderen können dann von einer korrekten Lösung profitieren.
\end{solution}

\begin{question}{5}
    Beschreiben Sie das Modell des differenziellen Lernens.
\end{question}
\begin{solution}
    Frage noch nicht bearbeitet. Bitte melde dich in einer Mail (Mailadresse auf dem Titelblatt) oder nutze git, wenn du eine Lösung für die Frage hast. Alle anderen können dann von einer korrekten Lösung profitieren.
\end{solution}

\begin{question}{2}
    Definieren Sie den Begriff Techniktraining.
\end{question}
\begin{solution}
    Frage noch nicht bearbeitet. Bitte melde dich in einer Mail (Mailadresse auf dem Titelblatt) oder nutze git, wenn du eine Lösung für die Frage hast. Alle anderen können dann von einer korrekten Lösung profitieren.
\end{solution}

\begin{question}{4}
    Beschreiben Sie den Unterschied zwischen Ziel- und Handlungsorientierung.
\end{question}
\begin{solution}
    Frage noch nicht bearbeitet. Bitte melde dich in einer Mail (Mailadresse auf dem Titelblatt) oder nutze git, wenn du eine Lösung für die Frage hast. Alle anderen können dann von einer korrekten Lösung profitieren.
\end{solution}

\begin{question}{6}
    Nennen Sie vier Grundsätze der Methodik im Techniktraining aus Sicht des DSV und erläutern Sie diese.
\end{question}
\begin{solution}
    Frage noch nicht bearbeitet. Bitte melde dich in einer Mail (Mailadresse auf dem Titelblatt) oder nutze git, wenn du eine Lösung für die Frage hast. Alle anderen können dann von einer korrekten Lösung profitieren.
\end{solution}

\begin{question}{6}
    Worauf müssen Sie bei der Rückmeldung an Ihren Schüler achten? Geben Sie drei Beispiele und erläutern Sie diese.
\end{question}
\begin{solution}
    Frage noch nicht bearbeitet. Bitte melde dich in einer Mail (Mailadresse auf dem Titelblatt) oder nutze git, wenn du eine Lösung für die Frage hast. Alle anderen können dann von einer korrekten Lösung profitieren.
\end{solution}